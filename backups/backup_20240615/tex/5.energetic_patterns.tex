\chapter{ Southwestern Atlantic Cyclones Energetics} \label{energetic_patterns}


\begin{itemize}
    \item Aqui, estou na dúvida se não seria bom criar um capítulo (ou uma seção?) para falar do ciclo de vida dos ciclones, visto que os sistemas serão normalizados a partir disso, havendo diferentes configurações de ciclos de vidas, por exemplo. 
    \item No caso, pensei em criar um capítulo para o ciclo de vida, apresentando os mapas espaciais de modo a indicar que a metodologia utilizada é válida e corresponde ao esperado pela literatura de climatologia de ciclones (ao mesmo tempo em que adiciona novas informações). Entretanto poderia acabar ficando desconexo com os objetivos da tese. 
    \item A alternativa seria apenas incluir como uma seção neste capítulo aqui, de modo que eu indique apenas as configurações de ciclones detectadas pelo programa e mostre alguns exemplos para aferir confiabilidade aos resultados, mas poderia acabar ficando desconexa das outras seções.
\end{itemize}

\section{Características gerais}
\begin{itemize}
    \item     Estatísticas gerais da energética dos sistemas
    \item     Compósitos para alguns termos, para diferentes fases do ciclo de vida
    \item Mostrar diagrama de fase para todos os casos

\end{itemize}

\section{Padrões energéticos}
\begin{itemize}
    \item Resultados das componentes principais
    \item Clusters identificados pelo K-means
    \item Resultados dos padrões energéticos

\end{itemize}

\section{Limitações, aplicações e passos futuros}
\begin{itemize}
    \item Limitações: metodologia semi-lagrangiana deve ser interpretada como snapshots (relacionando com \citet{muench1965dynamics})
    \item  A formulação adotada apenas permite a seguinte interpreteção: contribuição para energética global e não a energética individual de cada sistema
    \item Contextualizar a energética como ferramenta para determinação objetiva das causas eficientes e finais dos ciclones (diagramas do Hart estão relacionados com causas formais e materiais - os trabalhos complementam-se)
    \item Estudos de caso? (e.g. ciclones extratropicais clássicos formados no sul da ARG, ciclones bomba formados em LA-PLATA, ciclones subtropicais formados em SE-BR e, ciclones tropicais Anita, Iba, 01Q.
\end{itemize}