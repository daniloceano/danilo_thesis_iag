\chapter{Conclusions}\label{ch:conclusion}

The main goal of this thesis was to determine the primary dynamical mechanisms related to cyclone development in the Southwestern Atlantic and Southeast American regions during each of their development phases. For this, a database of cyclonic system tracks with genesis in this region was utilized. This database was constructed by detecting cyclones from their central relative vorticity at 850 hPa and includes all systems in the South Atlantic region from 1979 to 2020. Subsequently, a program called CycloPhaser was created to dissect the relative vorticity series into distinct life phases, based on its maximum, minimum, and derivative values. The CycloPhaser was then used to construct a climatology of cyclone life cycle phases from the original track database, which also allowed for investigating the mean characteristics of cyclones across each phase.

The next step involved programming an application to compute the Lorenz Energy Cycle and using it to analyze the energetics of all cyclones in the database. The mean characteristics of the systems' energy cycles were investigated and dissected for each life cycle phase. Afterward, the K-means algorithm was employed to determine the energy patterns. Finally, the dynamical mechanisms related to cyclonic system development, as indicated by the energy cycle analysis, were further investigated.

\section{Summary of Main Findings}

This thesis provides significant insights into the detection, analysis, and energetics of 7,531 cyclonic systems in the Southwestern Atlantic region through the development and application of new methodologies.

First, the CycloPhaser's ability to detect and analyze cyclone life cycles was demonstrated, highlighting key patterns in cyclone behavior. The analysis confirmed known cyclogenesis regions, cyclone paths, and statistical properties such as average lifetime and displacement, providing new insights related to each development phase. It was revealed that cyclonic systems exhibit multiple life phase configurations, with some configurations including secondary life cycle development. The analysis of spatial distributions across various phases of the cyclone life cycle reveals that systems often intensify and mature near their genesis regions, decaying over a broader area.

Second, the thesis delved into the Lorenz Energy Cycle (LEC) to assess the energetics of the selected cyclonic systems in the Southwestern Atlantic. Using a Semi-Lagrangian approach, the study revealed high variability in LEC terms, highlighting the importance of the baroclinic chain, convective activity, and barotropic conversions in cyclone development. The main mode of variability emphasized the significance of these processes throughout the cyclone life cycle, from cyclogenesis to decay, with the moist baroclinic chain peaking during the intensification phase and the barotropic conversions peaking during the mature phase. Nevertheless, these conversions continue even during the decay phase, where exports and especially dissipation of eddy kinetic energy lead to the eventual dissipation of the eddy. Furthermore, there is a relevant group of systems for which the imports of eddy APE are important during the cyclones' initial life stages.

Third, the Lorenz Phase Space (LPS) diagrams were introduced as a diagnostic tool for eddy-related energetics. These diagrams provided a visual representation of the dynamic mechanisms in cyclone development, including baroclinic and barotropic conversions and imports of eddy kinetic and available potential energy. The analysis identified three energy patterns (EPs) with distinct energy states and vorticity values, providing new perspectives on cyclone energetics. Furthermore, the occurrence of barotropic and baroclinic instabilities was further indicated through the application of the Rayleigh-Kuo criterion and the analysis of the maximum Eady Growth Rate for composites and case studies.

\section{Final Remarks and Recommendations for Future Work}

The Lorenz Energy Cycle of cyclonic systems in the Southwestern Atlantic and Southeast America region indicates that barotropic instability plays a pivotal role in extratropical cyclone development, which is commonly associated with baroclinic instability. Although the occurrence of such instabilities was not directly assessed, the indications from the energy cycle and the Rayleigh-Kuo criterion provide strong evidence for this argument. It is also demonstrated that the use of the Fixed framework for computing the Lorenz Energy Cycle underestimates the role of barotropic conversions while capturing baroclinic effects unrelated to cyclone development, which might produce misleading results. 

The findings presented here lead to the following research questions, left as suggestions for future work:

\begin{itemize}
    \item Why do cyclones have distinct life cycle configurations? For example, why do some systems not present an incipient stage, and why do others present a secondary life cycle? To answer these questions, the energy cycle (as done in the current study) as well as vorticity and heat budgets \citep[e.g.]{dutra2017structure} can be employed, and the different life cycle configurations can be assessed.
    \item How does climate change affect the spatial distribution of cyclone phases in the South Atlantic region? Will cyclones mature closer to the coast? Will phase characteristics (duration, displacement, etc.) change? This could be evaluated using climate models, as has already been done in the literature \citep[e.g.]{reboita2018extratropical,de2022future}, but using the CycloPhaser program to diagnose differences in life cycle phases.
    \item Which terms of the barotropic conversion term $C_K$ present the dominant magnitude for the cyclones in the current dataset? Do their behaviors change across distinct life cycles? This could be evaluated by modifying the Lorenz-cycle program to export the results for each distinct term and comparing the results.
    \item What are the actual physical mechanisms related to the imports of eddy APE and kinetic energy? To investigate the former, vorticity and heat budgets can be employed \citep[e.g.]{dutra2017structure}, while for the latter, an initial point could be investigating whether it can be attributed to downstream development \citep[e.g.]{piva2010energetics}.
\end{itemize}