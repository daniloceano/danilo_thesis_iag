Cyclones are critical elements of the Earth's climate system, significantly influencing weather patterns and climatic conditions while posing risks to life, property, and ecosystems. Understanding these systems is vital for enhancing weather forecasting, disaster preparedness, and mitigating their adverse effects, particularly in South America, where they affect precipitation regimes, cause extreme weather events, and impact various socio-economic sectors. Utilizing the Lorenz Energy Cycle (LEC) framework, this study aims to determine the main dynamical mechanisms driving cyclone development across various phases of their life cycle. Cyclone track data from 1979 to 2020, sourced from the ERA5 reanalysis and generated using the TRACK algorithm to identify cyclones based on central relative vorticity at 850 hPa, was used in this study. The CycloPhaser program was developed to categorize the vorticity time series into distinct life phases, providing a detailed climatology of cyclone life cycles. The LEC was computed for these cyclones to analyze their mean energy characteristics during each development phase using a Semi-Lagrangian Framework. Clustering analysis, empirical orthogonal functions, and statistical methods were utilized to analyze and characterize the energy patterns and dynamical mechanisms of these systems. The analysis reveals that cyclonic systems exhibit multiple life phase configurations, with some configurations including secondary life cycle development. The spatial distributions show that systems often intensify and mature near their genesis regions and decay over a broader area, providing new insights into South Atlantic storm tracks and cyclone development. The results reveal significant variability in the LEC terms across different cyclonic systems, highlighting the role of both baroclinic and barotropic instabilities in cyclone development. These processes peak during the intensification and mature phases and continue into the decay phase, underscoring their role throughout the cyclone life cycle.  The Lorenz Phase Space diagrams provide a novel visualization tool, facilitating the understanding of environmental energetics related to cyclonic systems. The analysis identifies distinct energetic patterns for various life cycle configurations, emphasizing the significance of barotropic conversions and convective activity. The findings suggest that enhanced convective activity and energy conversions are critical in certain life cycle configurations, contributing to a deeper understanding of cyclone dynamics in the Southwestern Atlantic. The occurrence of barotropic and baroclinic instabilities was further confirmed through the Rayleigh-Kuo criterion and by analyzing the maximum Eady Growth Rate. The results indicate that barotropic instability is more pronounced on smaller spatial scales and therefore was not entirely noted by classical Fixed Framework studies. Furthermore, this study presents a pioneering comprehensive climatology of cyclonic systems energetics, especially using a Semi-Lagrangian Framework, providing novel insights into the dynamics of these systems.
