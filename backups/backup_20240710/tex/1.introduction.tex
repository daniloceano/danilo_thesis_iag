\chapter{Introduction}
\label{intro}

Cyclones are an important part of the Earth's climate system. These systems are characterized by low atmospheric pressure at their center and rotating winds. Cyclones can be classified into different types, including tropical cyclones, extratropical cyclones, and subtropical cyclones, each distinguished by their physical characteristics. Cyclones play a significant role in the redistribution of heat and moisture across the Earth's surface, impacting weather patterns and climatic conditions. Additionally, cyclones can lead to severe weather events such as heavy rainfall, strong winds, and storm surges, posing substantial risks to life, property, and ecosystems. Understanding cyclones is vital for improving weather forecasting, disaster preparedness, and mitigating the adverse effects of these systems.


\section{Motivation}

Surface cyclones are crucial features for weather and climate in South America and the Southwestern Atlantic sector. They significantly influence the precipitation regimes on the continent, impacting both seasonal and annual rainfall patterns \citep{reboita2010regimes, reboita2018extratropical}. These cyclones can potentially cause extreme precipitation events, leading to flooding and landslides that can disrupt communities and infrastructure \citep{de2021ocean,de2024extreme}. Furthermore, the occurrence of cyclones is related to natural hazards along the South American coast due to their extreme winds \citep{de2021ocean,cardoso2022synoptic}, which can cause substantial damage to buildings, power lines, and other structures. High sea waves generated by cyclones \citep[e.g.]{guimaraes2014analysis,gramcianinov2023impact} pose risks to maritime activities and can affect shipping routes and port operations. The potential for cyclones to cause storm surges \citep[e.g.]{campos2010characterizatio,albuquerque2018determining,leal2023identification} can lead to significant coastal flooding, endangering coastal communities and ecosystems, while also contributing to coastal erosion \citep[e.g.]{parise2009extreme} that can undermine coastal defenses and infrastructure. Overall, the economic costs of cyclones in South America can be substantial, affecting livelihoods, infrastructure, and economic stability across various sectors. This reinforces the necessity of a detailed understanding of the mechanisms related to their genesis and intensification so that this knowledge can be incorporated into numerical models, ultimately improving forecasts.

\section{Scientific goals}

Given the importance of cyclonic systems for weather and climate in the South Atlantic sector, it is crucial to ensure accurate and precise forecasts of these systems. Recent studies have investigated cyclonic systems' projections for this region, primarily focusing on climatological trends \citep[e.g.]{reboita2018extratropical,de2022future}. How cyclone dynamics will change under climate change scenarios, especially in the South Atlantic region, remains an open question. However, a deep understanding of their current dynamics is necessary first. Although recent investigations have advanced the understanding of cyclone genesis and development in the South Atlantic \citep[e.g.]{dias2011energy,dias2013synoptic,gozzo2014subtropical,reboita2018key,gramcianinov2019properties}, most studies focus on case studies, and a climatological view of the dynamic mechanisms is still lacking. Furthermore, current cyclone climatologies offer only a unified view of the cyclone life cycle, not allowing for the geographical positioning and dynamical mechanisms of each phase to be investigated. Given this context, the main goal of the current thesis is to \textbf{determine the main dynamical mechanisms related to cyclone development in each of their development phases}. For this, the Lorenz Energy Cycle will be employed. Secondary scientific goals include:

\begin{itemize}
    \item Devise a climatology for each cyclone life phase
    \item Investigate the mean spatial distribution for each development phase
    \item Create a climatology of the energetics of cyclonic systems in the Southern Atlantic and Southeast South America regions
    \item Define patterns in the energy climatology
    \item Reveal the main energy fluxes related to cyclonic development in each life cycle phase
\end{itemize}

Chapter 2 is dedicated to the literature review, discussing cyclone characteristics, properties, development mechanisms, and their relationship with the atmospheric general circulation. It also includes an overview of methods for analyzing the cyclone life cycle and the state of the art of cyclone climatology for the South Atlantic region. It finishes with a thorough review of the Lorenz Energy Cycle methodology, its mathematical formulation, and a review of studies that employ this method for cyclonic systems. Chapter 3 presents the databases used in the current research, including the programs for detecting cyclone life cycles and computing the energy cycle, as well as the analysis procedures employed in this study. Chapter 4 introduces a new climatology of cyclones in the South Atlantic region, dissecting cyclones for distinct development phases. Part of the results presented in this chapter are published in \citet{deSouza2024}. Chapter 5 discusses the LEC for all systems with genesis in the study region, examining both their mean behavior and each development phase. Chapter 6 presents the Energy Patterns found for these systems and explores the dynamical mechanisms related to their energy cycle. Finally, Chapter 7 summarizes all results and presents the final remarks and conclusions of this thesis.
